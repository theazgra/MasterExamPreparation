\section{Matematika}
\url{http://www.studopory.vsb.cz/studijnimaterialy/Sbirka_uloh/video/obsah.html}
%---------------------------------------------------------------------------------------------
\subsection{Řešení soustav lineárních rovnic}
V matematice a lineární algebře se jako soustava lineárních rovnic označuje množina lineárních rovnic. Například:
\begin{equation}
\begin{aligned}
	3x_1 + 2x_2 - x_3 &= 1 \\
	2x_2 - 2x_x + 4x_3 &= -2 \\
	-x_1+\frac{1}{2}x_2-x_3 &=0
\end{aligned}
\end{equation}
\noindent Úkolem při řešení je najít takové hodnoty $x_1$, $x_2$ a $x_3$ pro které platí všechny rovnice zároveň.
Řešení soustavy se zapíše pomocí matic $A\cdot \vec{x} = \vec{b}$, $A$ je matice soustavy, $\vec{x}$ je vektor neznámých a $\vec{b}$ je vektor pravých stran, vzorec podrobně:
\begin{equation}
\left(
\begin{array}{c c c c}
a_{11} & a_{12} & \cdots & a_{1n}\\
a_{21} & a_{22} & \cdots & a_{2n}\\
\vdots & \vdots & \ddots & \vdots\\
a_{m1} & a_{m2} & \cdots & a_{mn}\\
\end{array}
\right)
\cdot
\left(
\begin{array}{c}
x_1\\
x_2\\
\vdots \\
x_n
\end{array}
\right)
=
\left(
\begin{array}{c}
b_1\\
b_2\\
\vdots \\
b_n
\end{array}
\right)
\end{equation}

\textbf{Řešení pomocí inverzní matice} $A^{-1}$, jde použít pokud $A$ je čtvercová matice a $A^{-1}$ existuje.
\begin{equation}
\vec{x} = A^{-1}\cdot \vec{b}
\end{equation}

\textbf{Gaussova eliminační metoda}
Příklad pro matici $A$ s rozměry $2\times 2$. Tuto matici musíme upravit do schodového tvaru, tzn. každý řádek má o jednu nulu navíc než předchozí
\begin{equation}
\left(
\begin{array}{c c | c}
a_{11} & a_{12} & b_1 \\
a_{21} & a_{22} & b_2
\end{array}
\right)
\end{equation}

\textbf{Gauss--Jordanova metoda} Je jinak zapsané řešení pomocí inverzní matice. Matici $A$ dáme do tvaru jako v Gaussově eliminační metodě a upravujeme levou část na jednotkovou matici.

%---------------------------------------------------------------------------------------------
\subsection{Vektorový prostor}
Vektorový (lineární) prostor je neprázdná množina $V$, jejíž prvky nazýváme vektory. V tomto prostoru jsou definovány 2 operace. Sčítání, tj. zobrazení $V \times V \rightarrow V$, značíme $+$ a násobení vektoru reálným číslem $\cdot$. Tyto dvě operace musí splňovat tyto podmínky pro, $\forall \vec{x},\vec{y},\vec{z} \in V$ a $\forall a,b \in \R$:
\begin{enumerate}
\item Sčítání vektorů je komutativní $\vec{x} + \vec{y} = \vec{y}+\vec{x}$
\item Sčítání vektorů je asociativní $(\vec{x}+\vec{y})+\vec{z} = \vec{x}+(\vec{y}+\vec{z})$
\item Asociativní zákon násobení $a\cdot(b\cdot\vec{x}) = (a\cdot b)\cdot\vec{x}$
\item Distributivní zákon pro vektory $a\cdot(\vec{x}+\vec{y}) = a\cdot\vec{x} + a\cdot\vec{y}$
\item Distributivita sčítání čísel $(a+b)\cdot \vec{x} = a\cdot \vec{x} + b\cdot \vec{x}$
\item Výsledkem násobení vektoru číslem 1 je stejný vektor $1 \cdot \vec{x} = \vec x$
\item Existuje nulový vektor, značí se tučnou nulou $\textbf{0}$, získáme jej vynásobením vektoru 0.
\item Skalární součin $x,y\in \R^2 : \vec{x}\cdot\vec{y} = x_1 y_1 + x_2 y_2$
\end{enumerate}
%---------------------------------------------------------------------------------------------
\subsection{Lineární zobrazení}
Pojmem lineární zobrazení (lineární transformace) se v matematice označuje takové zobrazení mezi vektorovými prostory $X$ a $Y$, které zachovává vektorové operace sčítání a násobení skalárem.

Nechť $U$ a $V$ jsou vektorové prostory, zobrazení $A : U \rightarrow V$ se nazývá lineární zobrazení, jestliže, pro každé dva vektory $u,v\in U$ a skalár $\alpha$ platí:
\begin{enumerate}
\item $A(u+v)= A(u) + A(v)$
\item $A(\alpha u) = \alpha A(u)$
\end{enumerate}

Nulový prostor zobrazení (Vektorové prostory $A,B$, zobrazení $A : U \rightarrow V$) je množina vzorů $\textbf{0}$, tj. $N(A)  = \{ u\in U: A(u)=\textbf{0} \}$.

%---------------------------------------------------------------------------------------------
\subsection{Derivace reálné funkce}
Derivací funkce získáme směrnici tečny.
\begin{table}[h!]
\centering
\begin{tabular}{l l || l l}
\hline
\textbf{Funkce} & \textbf{Derivace} & \textbf{Funkce} & \textbf{Derivace}\\
\hline\hline
$const'$ & 0 & $x'$ & 1\\
$(x^c)'$ & $cx^{c-1}$ & $(f+g)'$ & $f' + g'$\\
$(const \cdot f)'$ & $const \cdot f'$ & $(f\cdot g)'$ & $f'\cdot g + f\cdot g'$\\
$\left(\frac{f}{g}\right)'$ & $\frac{f'\cdot g - f\cdot g'}{g^2}$ & $(f(g(x)))'$ & $f'(g(x))\cdot g'(x)$\\
$(c^x)'$ & $c^{x \ln c}: c>0$ & $(e^x)'$ & $e^x$\\
$(\log_c x)'$ & $\frac{1}{x \ln c} : c>0\wedge c\neq 0$ & $(\ln x)'$ & $\frac{1}{x}$\\
$(\sin x)'$ & $\cos x$ & $(\cos x)'$ & $-\sin x$\\
$(\tan x)'$ & $\frac{1}{\cos^2 x}$ & $(\cot x)'$ & $-\frac{1}{\sin^2 x}$\\
$(\arcsin x)'$ & $\frac{1}{\sqrt{1-x^2}}$ & $(\arccos x)'$ & $-\frac{1}{\sqrt{1-x^2}}$\\
$(\arctan x)'$ & $\frac{1}{1+x^2}$ & $(\text{arccot } x)'$ & $-\frac{1}{1+x^2}$\\
\hline
\end{tabular}
\end{table}

%---------------------------------------------------------------------------------------------
\subsection{Určitý a neurčitý integrál}
\subsubsection{Určitý integrál}
Výpočet plochy, která je vymezena grafem funkce $f$ na $<a,b>$ a osou x.
\begin{equation}
\int_a^b f(x) dx
\end{equation}
\subsubsection{Neurčitý integrál}
Určení funkce, je-li známa její derivace $\rightarrow$ hledání primitivní funkce. Pojmem integrál se občas označuje primitivní funkce $F$, jejíž derivací je funkce $f$. To celé se pak nazývá neurčitý integrál a zapisuje se jako:
\begin{equation}
 F = \int f(x) dx
\end{equation}

%---------------------------------------------------------------------------------------------
\subsection{Kombinatorické výběry}
Uspořádané výběry jsou permutace a kombinace, záleží na pořadí vybraných prvků. Naopak variace je neuspořádaný výběr. Výběry s opakováním, prvky mohou být do výběry zařazeny opakovaně.
Kombinační číslo:
\begin{equation}
\binom{n}{k} = \frac{n!}{(n-k)!\cdot k!}
\end{equation}
\subsubsection{Variace bez opakování}
Variace bez opakování $k$-té třídy z $n$ prvků nazýváme každou uspořádanou $k$-prvkovou podmnožinu $n$ prvkové základní množiny $M$.
\begin{equation}
V(n,k) = \frac{n!}{(n-k)!}
\end{equation}
\subsubsection{Variace s opakováním}
\begin{equation}
V^*(n,k) = n^k	
\end{equation}
\subsubsection{Kombinace bez opakování}
\begin{equation}
C(n,k) = \frac{n!}{(n-k)!\cdot k!}
\end{equation}
\subsubsection{Kombinace s opakováním}
\begin{equation}
C^*(n,k) = \frac{(n+k-1)!}{(n-1)!\cdot k!}
\end{equation}
\subsubsection{Permutace bez opakování}
\begin{equation}
P(n) = n!
\end{equation}
\subsubsection{Permutace s opakováním}
\begin{equation}
P^*(n_1, n_2, \ldots n_k) = \frac{n!}{n_1!+n_2!+\ldots n_k!}
\end{equation}
%---------------------------------------------------------------------------------------------
\subsection{Grafy a jejich využití}
Orientovaný graf $G$ je dvojice ($V$, $E$), kde $V$ je konečná množina vrcholů a $E \subseteq V\times V$ je množina hran.
Neorientovaný graf se znázorňuje podobně jako orientovaný, na konci čar, které představují hrany však nekreslíme šipky.
(\textit{V některých definicích neorientovaného grafu nejsou povoleny smyčky.})

Vizualizace informací, hledaní cesty v grafu (navigace, routování path finding), proudění v sítích.